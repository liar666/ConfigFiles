\documentclass[french,t,% 't' (resp. 'c') places text vertically at top/center of the slides
% PDF settings
hyperref={%
	pdftitle={...},%
	pdfauthor={...},%
	pdfsubject={...},%
	pdfkeywords={...}%
      },%
xcolor={pdftex,svgnames}, % dvipsnames, dvipsnames*, svgnames, svgnames*, x11names,
]{beamer}  %
\usetheme{default} % AnnArbor,Antibes,Bergen,Berkeley,Berlin,Boadilla,CambridgeUS,Copenhagen,Darmstadt,Dresden,Frankfurt,Goettingen,Hannover,Ilmenau,JuanLesPins,Luebeck,Madrid,Malmoe,Marburg,Montpellier,PaloAlto,Pittsburgh,Rochester,Singapore,Szeged,Warsaw,boxes,default

% Correct French/English indentation and splitting of words
\usepackage{babel}

% Correct management of accentuated chars in input file
\usepackage[latin1]{inputenc}
%\usepackage[utf8]{inputenc}

% Correct font for the generation of docs with accentuated chars
\usepackage[T1]{fontenc}      % Can handle hyphenation of words with accented characters
%%\usepackage[OT1]{fontenc}   % Might generated bad looking PDFs

% Access to many maths symbols
\usepackage{amsthm}
\usepackage{amsmath}
\usepackage{amsfonts}

% Insertion of images generated by external tools 
\usepackage{graphicx}

% To generate pretty & scalable images directly in LaTeX 
\usepackage{tikz}
% \draw[decorate,decoration={coil,amplitude=1.5cm, segment length=.4cm}] (5,5.5) -- (5,1.5) ;
\usetikzlibrary{decorations.text}
\usetikzlibrary{decorations.shapes} 
\usetikzlibrary{decorations.pathmorphing,snakes}

% To print numbers correctly
\usepackage{numprint}



% Info for title page
\title[<short>]{...}
\author[<short>]{... \and ... \inst{2}}
\institute[<short>]{
  \inst{1}%
  ...
  \and
  \inst{2}%
  ...}
\date[<short>]{\today{}}
\subject{...}


%%%%%%%%%%%%%%%%%%%%%%%%%%%%%%%%%%%%%%%%%%%%%%%%%%%%%%%%%%%%%%%%%%%%%%
\begin{document}

% The title page
\begin{frame}
  \titlepage
\end{frame}

% The TOC
\begin{frame}{Table of contents}
  \tableofcontents
\end{frame} 

%%%%%%%%%%%%%%%%%%%%%%%%%%%%%%%%%%%%%%%%
\part{}
\section{}
\subsection{}


% First slide
\begin{frame}{...}

\only<1-2>{...}  % Shows the elements greyed when slide(s) is not the one indicated 

\uncover<3->{...} % Only shows the elements when slide(s) is the one indicated

\begin{figure}
  \begin{center}
    \includegraphics[width=0.65\textwidth]{...}
  \end{center}
\end{figure}

\begin{block}{Un bloc normal}
  Texte du block \texttt{block}
 \end{block}

 \begin{alertblock}{Un bloc tr�s alerte}
  Texte du block \texttt{alertblock}
 \end{alertblock}

 \begin{exampleblock}{Un bloc exemplaire}
  Exemple de block \texttt{exampleblock}

 \end{exampleblock}
  \begin{definition}
    environnement definition
  \end{definition}
  
 \begin{example}
   environnement example
 \end{example}

\begin{proof}
   environnement proof
 \end{proof}  
    
\begin{theorem}
environnement theorem
\end{theorem}

\end{frame}

\begin{columns}[options] % 't' (resp. 'c') places text vertically at top/center of the slides
  \begin{column}[t]{0.40\textwidth}
    contenu
  \end{column}
  \begin{column}[c]{0.59\textwidth}
    contenu
  \end{column}
\end{columns}


\end{document}


%%% Local Variables: 
%%% mode: latex
%%% TeX-master: t
%%% End: 
